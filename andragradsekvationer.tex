\section{Andragradsekvationer}

Ett vanligt användningsområde för funktioner är, som tidigare sagt, att modellera samband -- exempelvis hur busspriser påverkar resvanor, hur körhastighet påverkar bensinförbrukning, och så vidare.
Vitsen med att sätta upp funktionsuttryck för att beskriva de sambanden är att vi kan räkna på, säg, ett visst pris på bussbiljetter påverkar resvanorna, utan att behöva genomföra många dyra experiment.
Utan att testa hundra olika biljettpriser kan vi hitta det pris som, förmodligen, ger största möjliga vinst för bussbolaget, och vi kan också förutsäga hur många som kommer att åka buss vid olika biljettpriser.

Så länge vi har ett funktionsuttryck är det ganska lätt att gå från ett $x$-värde till ett värde på $f(x)$ -- metoden är som vanligt att sätta in $x$-värdet och räkna ut vad $f(x)$ blir.
Men ibland vill vi det omvända: vi vill ta reda på vilket $x$-värde som ger exempelvis $f(x)=200$.
Det leder oss till \emph{andragradsekvationer}, det vill säga ekvationer som innehåller andragradsuttryck.
Vi kommer att märka att de är betydligt jobbigare att lösa än våra vanliga (``linjära'') ekvationer, men vi kommer att lära oss en metod för att lösa dem -- plus en smart genväg för att lösa några speciella andragradsekvationer.

\subsubsection{Ett exempel på en andragradsekvation}

\framebox{\parbox{
  \textbf{Exempelproblem}
  I Tomellilla en dag i juni kunde temperaturen beskrivas med funktionen $f(t) = -0,1(x-13)^2+28$. //
  $t$: antalet timmar sedan midnatt
  $f(t)$: temperaturen i grader celsius
  
  \begin{enumerate}[label=\Alph*]
    \item Hur varmt var det kl. 10?
    \item När var det som varmast, och hur varmt var det då?
    \item När var det 18°C?
  \end{enumerate}
  
  \textbf{Lösningsförslag}
  \begin{enumerate}[label=\Alph*]
    \item Temperaturen 10 timmar efter midnatt ges av $f(10)$, som vi beräknar till $-0,1(10-13)^2+28 = -0,1 \cdot 3^2 + 28 = -0,9+28 = 27,1$. //
    Svar: Kl. 10 var temperaturen var 27,1°C.
    \item Genom att titta på funktionsuttrycket kan vi se att extremvärdet är 28, och det får vi när $x=13$. //
    Svar: Det var som varmast kl. 13, och då var det 28°C.
    \item Att det är 18°C är samma sak som att säga $f(x)=18$, vilket ger oss ekvationen $-0,1(x-13)^2+28=18$. //
    \emph{Hur löser vi den?}
  \end{enumerate}
}}

\subsubsection{Den enklaste andragradsekvationen}

