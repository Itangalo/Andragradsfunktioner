\section{Andragradsekvationer}

Ett vanligt användningsområde för funktioner är, som tidigare sagt, att modellera samband -- exempelvis hur busspriser påverkar resvanor, hur körhastighet påverkar bensinförbrukning, och så vidare.
Vitsen med att sätta upp funktionsuttryck för att beskriva de sambanden är att vi kan räkna på, säg, ett visst pris på bussbiljetter påverkar resvanorna, utan att behöva genomföra många dyra experiment.
Utan att testa hundra olika biljettpriser kan vi hitta det pris som, förmodligen, ger största möjliga vinst för bussbolaget, och vi kan också förutsäga hur många som kommer att åka buss vid olika biljettpriser.

Så länge vi har ett funktionsuttryck är det ganska lätt att gå från ett $x$-värde till ett värde på $f(x)$ -- metoden är som vanligt att sätta in $x$-värdet och räkna ut vad $f(x)$ blir.
Men ibland vill vi det omvända: vi vill ta reda på vilket $x$-värde som ger exempelvis $f(x)=200$.
Det leder oss till \emph{andragradsekvationer}, det vill säga ekvationer som innehåller andragradsuttryck.
Vi kommer att märka att de är betydligt jobbigare att lösa än våra vanliga (``linjära'') ekvationer, men vi kommer att lära oss en metod för att lösa dem -- plus en smart genväg för att lösa några speciella andragradsekvationer.

\subsection{Ett exempel på en andragradsekvation}

\textbf{Exempelproblem}
I Tomellilla en dag i juli kunde temperaturen beskrivas med funktionen $f(t) = -0,1(x-13)^2+28$.
\begin{itemize}
  \item $t$: antalet timmar sedan midnatt
  \item $f(t)$: temperaturen i grader celsius
\end{itemize}

Tobias gillar att springa, men vill inte springa när det är varmare än 18^{\circ}C.

\begin{enumerate}[(a)]
  \item Hur varmt var det kl. 10?
  \item När var det som varmast, och hur varmt var det då?
  \item När var det 18$^{\circ}$C?
\end{enumerate}

\textbf{Lösningsförslag}
\begin{enumerate}[(a)]
  \item Temperaturen 10 timmar efter midnatt ges av $f(10)$, som vi beräknar till $-0,1(10-13)^2+28 = -0,1 \cdot 3^2 + 28 = -0,9+28 = 27,1$. \\
  \textbf{Svar:} Kl. 10 var temperaturen 27,1$^{\circ}$C.
  \item Genom att titta på funktionsuttrycket kan vi se att extremvärdet är 28, och det får vi när $x=13$. \\
  \textbf{Svar:} Det var som varmast kl. 13, och då var det 28$^{\circ}$C.
  \item Att det är 18$^{\circ}$C är samma sak som att säga $f(x)=18$, vilket ger oss ekvationen $-0,1(x-13)^2+28=18$. \\
  \textbf{Hur löser vi den?}
\end{enumerate}

\subsubsection{En mycket enklare andragradsekvation}

Ja, hur löser vi ekvationen $-0,1(x-13)^2+28=18$?

Låt oss börja med att titta på en mycket enklare ekvation: $-0,1t^2+28=18$.
Det här är en så kallad potensekvation, som du förmodligen sett en del av i matte 1-kursen.
Vi kan lösa den genom att isolera potensen ($t^2$) på ena sidan av ekvationen, och sedan ta kvadratroten ur båda leden:

\begin{tabular}{l|p{4.7cm}}
  $-0,1t^2+28=18$ & dra bort 28 \\
  $\Leftrightarrow -0,1t^2=-10$ & multiplicera med $-1$ \\
  $\Leftrightarrow 0,1t^2=10$ & dividera med $0,1$ \\
  $\Leftrightarrow t^2=100$ & dra kvadratroten ur båda leden och ta med negativ lösning \\
  $\Leftrightarrow t=10$ eller $t=-10$
\end{tabular}

Observera att vi även tar med den negativa lösningen i sista steget -- ekvationen $t^2=100$ har \emph{två} lösningar: $t=10$ och $t=-10$.

\subsubsection{Att lösa andragradsekvationer}

Vi kan lösa andragradsekvationer på precis samma sätt som potensekvationen ovan, om vi behandlar parentesen med $x$ som en enhet:

\begin{tabular}{l|p{4.7cm}}
  $-0,1(x-13)^2+28=18$ & dra bort 28 \\
  $\Leftrightarrow -0,1(x-13)^2=-10$ & multiplicera med $-1$ \\
  $\Leftrightarrow 0,1(x-13)^2=10$ & dividera med $0,1$ \\
  $\Leftrightarrow (x-13)^2=100$ & dra kvadratroten ur båda leden och ta med negativ lösning \\
  $\Leftrightarrow (x-13)=10$ eller $(x-13)=-10$
\end{tabular}

Den enda skillnaden är att vi blir tvungna att göra ett extra steg i slutet, där vi räknar ut vad $x$ är (istället för $x-13$):

\begin{tabular}{l|p{4.7cm}}
  (x-13)=10$ eller $(x-13)=-10$ & lägg till 13 \\
  $\Leftrightarrow x = 23$ eller $x=3$
\end{tabular}
