\section{Andragradsekvationer}

Ett vanligt användningsområde för funktioner är, som tidigare sagt, att modellera samband -- exempelvis hur busspriser påverkar resvanor, hur körhastighet påverkar bensinförbrukning, och så vidare.
Vitsen med att sätta upp funktionsuttryck för att beskriva de sambanden är att vi kan räkna på, säg, ett visst pris på bussbiljetter påverkar resvanorna, utan att behöva genomföra många dyra experiment.
Utan att testa hundra olika biljettpriser kan vi hitta det pris som, förmodligen, ger största möjliga vinst för bussbolaget, och vi kan också förutsäga hur många som kommer att åka buss vid olika biljettpriser.

Så länge vi har ett funktionsuttryck är det ganska lätt att gå från ett $x$-värde till ett värde på $f(x)$ -- metoden är som vanligt att sätta in $x$-värdet och räkna ut vad $f(x)$ blir.
Men ibland vill vi det omvända: vi vill ta reda på vilket $x$-värde som ger exempelvis $f(x)=200$.
Det leder oss till \emph{andragradsekvationer}, det vill säga ekvationer som innehåller andragradsuttryck.
Vi kommer att märka att de är betydligt jobbigare att lösa än våra vanliga (``linjära'') ekvationer, men vi kommer att lära oss en metod för att lösa dem -- plus en smart genväg för att lösa några speciella andragradsekvationer.

\subsection{Ett exempel på en andragradsekvation}

\textbf{Exempelproblem}

Tobias bor i Tomellilla, och gillar långdistandslöpning. Men han vill inte springa när det är varmare än 18$^{\circ}$C.
En dag i juli kunde temperaturen beskrivas med funktionen $f(x) = -0,1(x-13)^2+28$. (Se figur nedan.)
\begin{itemize}
  \item[] $t$: antalet timmar sedan midnatt
  \item[] $f(t)$: temperaturen i grader celsius
\end{itemize}

\begin{figure}
  \centering
  \includegraphics[width=0.7\textwidth]{bilder/temperatur.png}
  \caption{\label{fig:temperatur}Grafen till funktionen $f(x) = -0,1(x-13)^2+28$, med linjen $y=18$ markerad.}
\end{figure}

\begin{enumerate}[(a)]
  \item Hur varmt var det kl. 10 när Tobias hade gått upp och ätit frukost?
  \item När var det som varmast, och hur varmt var det då?
  \item När var det 18$^{\circ}$C?
\end{enumerate}

\textbf{Lösningsförslag}
\begin{enumerate}[(a)]
  \item Temperaturen 10 timmar efter midnatt ges av $f(10)$, som vi beräknar till $-0,1(10-13)^2+28 = -0,1 \cdot 3^2 + 28 = -0,9+28 = 27,1$. \\
  \textbf{Svar:} Kl. 10 var temperaturen 27,1$^{\circ}$C.
  \item Genom att titta på funktionsuttrycket kan vi se att extremvärdet är 28, och det får vi när $x=13$. \\
  \textbf{Svar:} Det var som varmast kl. 13, och då var det 28$^{\circ}$C.
  \item Att det är 18$^{\circ}$C är samma sak som att säga $f(x)=18$, vilket ger oss ekvationen $-0,1(x-13)^2+28=18$. \\
  Vi kan lösa ekvationen grafiskt, genom att hitta skärningspunkterna mellan $f(x)$ och linjen $y=18$.
  \textbf{Men hur löser vi ekvationen algebraiskt?}
\end{enumerate}

Ja, hur löser vi ekvationen $-0,1(x-13)^2+28=18$?

När vi har något i kvadrat är det lockande att testa att ta roten ur.
Det måste vi i så fall göra med \emph{hela} vänsterledet och \emph{hela} högerledet:
$\sqrt{-0,1(x-13)^2+28}=\sqrt{18}$

Här är det nödvändigt att komma ihåg att roten ur en summa inte (\emph{inte!}) är samma sak som roten ur varje term för sig:
$\sqrt{2+2} \neq \sqrt{2} + \sqrt{2}$
Därför kan vi inte förenkla vänsterledet i $\sqrt{-0,1(x-13)^2+28}=\sqrt{18}$ på något smidigt sätt.

Vad gör vi då?

\subsubsection{En mycket enklare andragradsekvation}

Låt oss börja med att titta på en mycket enklare ekvation: $-0,1t^2+28=18$.
Det här är en så kallad potensekvation, som du förmodligen sett en del av i matte 1-kursen.
Vi kan lösa den genom att isolera potensen ($t^2$) på ena sidan av ekvationen, och sedan ta kvadratroten ur båda leden:

\smallskip
\begin{tabular}{l|p{4.7cm}}
  $-0,1t^2+28=18$ & dra bort 28 \\
  $\Leftrightarrow -0,1t^2=-10$ & multiplicera med $-1$ \\
  $\Leftrightarrow 0,1t^2=10$ & dividera med $0{,}1$ \\
  $\Leftrightarrow t^2=100$ & dra kvadratroten ur båda leden och ta med negativ lösning \\
  $\Leftrightarrow t=10$ eller $t=-10$
\end{tabular}
\smallskip

Observera att vi även tar med den negativa lösningen i sista steget -- ekvationen $t^2=100$ har \emph{två} lösningar: $t=10$ och $t=-10$.

Vi kan lösa andragradsekvationer på precis samma sätt som potensekvationen ovan, om vi behandlar parentesen med $x$ som en enhet:
Den enda skillnaden är att vi blir tvungna att göra ett extra steg i slutet, där vi räknar ut vad $x$ är (istället för $x-13$):

\smallskip
\begin{tabular}{r p{5cm}|p{4.7cm}}
  & $-0,1(x-13)^2+28=18$ & dra bort 28 \\
  $\Leftrightarrow$ & $-0,1(x-13)^2=-10$ & multiplicera med $-1$ \\
  $\Leftrightarrow$ & $0,1(x-13)^2=10$ & dividera med $0,1$ \\
  $\Leftrightarrow$ & $(x-13)^2=100$ & dra kvadratroten ur båda leden och ta med negativ lösning \\
  $\Leftrightarrow$ & $x-13=10$ eller $x-13=-10$ & \textbf{lägg till 13} \\
  $\Leftrightarrow$ & $x = 23$ eller $x=3$  
\end{tabular}
\smallskip

Det är värt att poängtera att när vi drar kvadratroten ur $(x-13)^2$, är det \emph{hela} det uttrycket vi drar roten ur -- vi tar alltså inte roten ur $x$ och $13$ för sig.

\subsubsection{Två lösningar till andragradsekvationer}

När man löser andragradsekvationer kommer vi, i många lägen, få \emph{två} lösningar.
Om vi tittar på grafen ovan kan vi se att $f(x) = -0,1(x-13)^2+28$ får värdet 18 två gånger -- så det är inte konstigt att vi hittar två olika $x$-värden som löser ekvationen $f(x)=18$.
Båda dessa $x$-värden är korrekta lösningar, och det är ingen som är ``bättre'' eller ``mer korrekt'' än den andra.
Vi har helt enkelt två olika lösningar på ekvationen.
(I vårt fall betyder det att temperaturen blir 18$^{\circ}$C två gånger.)

När man har mer än en lösning till en ekvation är det vanligt att man numrerar de olika lösningarna.
I vårt fall hade vi skrivit $x_1=23$ och $x_2=3$.
Det nedsänkta talet kallas \emph{index}, och man uttalar lösningarna ``x-ett är 23 och x-två är 3''.
(De nedsänkta talen är alltså inget räknesätt, som det är med upphöjda tal, utan bara ett sätt att skilja olika $x$-värden åt.)

Ett annat sätt att markera två lösningar är att använda tecknet $\pm$.
Att skriva $x=\pm 5$ betyder att $x$ är $+5$ \emph{eller} $-5$.
I vår lösning ovan hade vi kunnat skriva $x=13 \pm 10$, vilket betyder att $x=13+10$ \emph{eller} $x=13-10$.
Om man väljer att använda $\pm$ eller skriva ut och numrera de olika lösningarna är en smaksak, men man bör tänka på att det ska vara enkelt att läsa.

\subsection{Fördjupning: Att göra substitutionen $t = x + d$}

(To be written.)
