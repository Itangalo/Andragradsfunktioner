\section{Att tolka andragradsuttryck}

Ekvationer för räta linjer kan skrivas på en rad olika sätt -- ekvationerna $y = -2x + 3$, $y + 2x = 3$ och $y + 2x - 3 = 0$ beskriver alla samma linje.
Beroende på hur man skriver ekvationen går det lättare eller svårare att läsa av olika egenskaper för linjen.
(Skriver vi på formen $y=-2x+3$ kan vi läsa av lutningen på linjen ($-2$) och skärningen med y-axeln ($3$), medan andra former har andra fördelar.)

Precis som räta linjer, kan andragradsfunktioner skrivas på ett antal olika sätt.
Det sätt man vanligtvis ser är $f(x) = ax^2 + bx + c$ (exempelvis $y=2x^2 - 12x + 19$), men vi kommer till en början att fokusera på formen $f(x) = a(x+d)^2+e$ (exempelvis $y=2(x-3)^2+1$).
Det sättet att skriva andragradsuttryck på kallas \textit{kvadratkompletterad form} (eftersom uttrycket består av en kvadrat som kompletterats med en konstant).

\subsection{Hitta extremvärdet för en andragradsfunktion}

asdf

\begin{desciption}
  \item[Allmän form]: $ax^2+bx+c$
  \item[Kvadratkompletterad form]: $a(x+d)^2+e$
\end{description}

