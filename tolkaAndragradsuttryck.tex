\section{Att tolka andragradsuttryck}

Ekvationer för räta linjer kan skrivas på en rad olika sätt -- ekvationerna $y = -2x + 3$, $y + 2x = 3$ och $y + 2x - 3 = 0$ beskriver alla samma linje.
Beroende på hur man skriver ekvationen går det lättare eller svårare att läsa av olika egenskaper för linjen, som lutning eller skärning med y-axeln.

Precis som räta linjer, kan andragradsfunktioner skrivas på ett antal olika sätt.
Det sätt man vanligtvis ser är \mbox{$f(x) = ax^2 + bx + c$}, exempelvis $y=2x^2 - 12x + 19$, men vi kommer till en början att fokusera på formen \mbox{$f(x) = a(x+d)^2+e$} istället -- exempelvis $y=2(x-3)^2+1$.
Det sättet att skriva andragradsuttryck på kallas \textit{kvadratkompletterad form} (eftersom uttrycket består av en kvadrat som kompletterats med en konstant).
Om en andragradsfunktion är skriven på kvadratkompletterad form är det lätt att hitta extrempunkter och extremvärden för funktionen, och det går också relativt lätt att lösa ekvationer som består av kvadratkompletterade andragradsuttryck.

\subsection{Hitta extremvärdet för en andragradsfunktion}

asdf

\begin{itemize}
  \item \textbd{Allmän form:} $ax^2+bx+c$
  \item \textbd{Kvadratkompletterad form:} $a(x+d)^2+e$
\end{itemize}
