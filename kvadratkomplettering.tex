\section{Att kvadratkomplettera andragradsuttryck}

Andragradsuttryck som är skrivna i kvadratkompletterad form har fördelen att vi enkelt kan läsa av extrempunkter och symmetrilinje, och vi kan dessutom lösa andragradsekvationer som är kvadratkompletterade.
Men i många lägen är inte andragradsuttryck skrivna i kvadratkompletterad form:
När vi utvecklar parentesmultiplikationer får vi andragradsuttryck skrivna på standardform ($ax^2+bx+c$), och det är också den vanligaste formen när vi får ``färdiga'' andragradsuttryck givna till oss.

I det här avsnittet ska vi lära oss hur vi omvandlar ett andragradsuttryck från standardform till kvadratkompletterad form.
Det finns flera metoder för det, och den vi kommer att använda kallas för \emph{ansätting}.
Det är en metod som är användbar i många olika delar av matematiken, och den låter oss också kvadratkomplettera uttryck utan att ta för många steg i huvudet.

\subsection{Uttrycket $a(x+d)^2+e$}

Låt oss börja kvadratkomplettera med ett exempel -- att kvadratkomplettera andragradsuttrycket $2x^2+28x+5$.

Spoiler: Kvadratkompletteringen kommer att leda till att vi hittar uttrycket $2(x+7)^2-93$.
Om vi testar att utveckla det uttrycket kommer vi att märka att det blir precis $2x^2+28x+5$ -- vilket är ett bra sätt att testa att kvadratkompletteringen är korrekt.

Att kvadratkomplettera uttrycket betyder att vi vill skriva om det till formen $a(x+d)^2+e$, och vår utmaning ligger i att hitta vilka värden på $a$, $d$ och $e$ som gör att vi får samma uttryck som vi började med.
Att testa om vår kvadratkomplettering stämmer är rätt lätt -- men hur går det till hitta värdena på $a$, $d$ och $e$?

Vår metod går ut på att ta ett uttryck $a(x+d)^2+e$, och steg för steg undersöka vad $a$, $d$ och $e$ måste ha för värden för att uttrycket ska vara samma sak som $2x^2+28x+5$.
Det kan vi göra genom att först utveckla $a(x+d)^2+e$, så att det blir skrivet i allmän form.

\smallskip
\begin{center}
\begin{tabular}{m{7cm}|m{5cm}|b{2cm}}
  $2x^2+28x+5 = a(x+d)^2+e$ & Vi \emph{ansätter} ett kvadratkompletterat andragradsuttryck, och påbörjar arbetet med att vilka värden $a$, $d$ och $e$ måste ha. \\
  \hline
  
  $\Leftrightarrow$
  \begin{tabular}{ l l l l }
    & $2x^2$ & $+28x$ & $+5$ \\
    $=$ & $ax^2$ & $+2adx$ & $+ad^2+e$ \\
  \end{tabular} & Vi utvecklar vårt ansatta uttryck, för att kunna jämföra term för term. & \\
  \hline

  $\Rightarrow$
  \begin{tabular}{ l l l l }
    & $\framebox{2}x^2$ & $+28x$ & $+5$ \\
    $=$ & $\framebox{a}x^2$ & $+2adx$ & $+ad^2+e$ \\
  \end{tabular} & När vi jämför termerna framför $x^2$ ser vi att $a$ måste vara 2 för att uttrycken ska vara lika. & $a=2$ \\
  \hline

  $\Rightarrow$
  \begin{tabular}{ l l l l }
    & $2x^2$ & $\framebox{+28}x$ & $+5$ \\
    $=$ & $2x^2$ & \framebox{$+2\cdot 2 \cdot d$}$x$ & $+2 \cdot d^2+e$ \\
  \end{tabular} & När vi jämför termerna framför $x$ får vi ekvationen $4d=28$, vilket ger $d = 7$. & $d=7$ \\
  \hline

  $\Rightarrow$
  \begin{tabular}{ l l l l }
    & $2x^2$ & $+28x$ & $\framebox{+5}$ \\
    $=$ & $2x^2$ &$+28x$ & \framebox{$+2 \cdot 7^2+e$} \\
  \end{tabular} & När vi jämför de konstanta termerna får vi ekvationen $2 \cdot 7^2 + e = 5$, vilket ger $e = 5 - 2 \cdot 49 = -93$. & $e=-93$ \\
  \hline

  $\Rightarrow 2x^2+28x+5 = 2(x+7)^2-93$
\end{tabular}
\end{center}
\smallskip

När vi kvadratkompletterat ett uttryck är det klokt att dubbelkolla att uttrycket stämmer, genom att utveckla det (med hjälp av kvadreringsregeln):

\begin{center}
  $2(x+7)^2-93 = 2 \cdot (x^2 + 14x + 49) - 93 = 2x^2 + 28x + 98 - 93 = 2x^2 + 28x + 5$
  
  Stämmer!
\end{center}

Vi tar ett exempel till:

\textbf{Exempeluppgift}
Kvadratakomplettera $3x^2 - 9x + 12$.

\smallskip
\begin{center}
\begin{tabular}{m{7cm}|m{5cm}|b{2cm}}
  $$3x^2 - 9x + 12$ = a(x+d)^2+e$ & Vi ansätter ett kvadratkompletterat andragradsuttryck och utvecklar det. \\
  \hline
  
  $\Leftrightarrow$
  \begin{tabular}{ l l l l }
    & $3x^2$ & $-9x$ & $+12$ \\
    $=$ & $ax^2$ & $+2adx$ & $+ad^2+e$ \\
  \end{tabular} & Vi börjar med att jämföra $x^2$-termerna. & \\
  \hline

  $\Rightarrow$
  \begin{tabular}{ l l l l }
    & $\framebox{3}x^2$ & $-9x$ & $+12$ \\
    $=$ & $\framebox{a}x^2$ & $+2adx$ & $+ad^2+e$ \\
  \end{tabular} & När vi har värdet på $a$ kan vi hitta värdet på $d$ genom att titta på $x$-termerna. & $a=3$ \\
  \hline

  $\Rightarrow$
  \begin{tabular}{ l l l l }
    & $3x^2$ & $\framebox{-9}x$ & $+12$ \\
    $=$ & $3x^2$ & \framebox{$+2\cdot 3 \cdot d$}$x$ & $+3 \cdot d^2+e$ \\
  \end{tabular} & Ekvationen $6d=-9$ ger oss värdet på $d$. Vi sätter in detta för att hitta värdet på $e$. & $d=\frac{-3}{2}$ \\
  \hline

  $\Rightarrow$
  \begin{tabular}{ l l l l }
    & $3x^2$ & $-9x$ & $\framebox{+12}$ \\
    $=$ & $3x^2$ &$-9x$ & \framebox{$+3 \cdot \left ( \frac{-3}{2} \right )^2+e$} \\
  \end{tabular} & Vi hittar värdet på $e$ genom ekvationen $3 \cdot \left ( \frac{-3}{2} \right )^2+e = 12$. & $e=\frac{21}{4}$ \\
  \hline

  $\Rightarrow 3x^2-9x+12 = 3(x+\frac{-3}{2})^2+\frac{21}{4}$
\end{tabular}
\end{center}
\smallskip
