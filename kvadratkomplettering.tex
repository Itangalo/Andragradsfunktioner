\section{Att kvadratkomplettera uttryck}

Metoden för att lösa andragradsekvationer som presenteras ovaan fungerar för att alla $x$-termer är samlade i en kvadrat, så som $(x - 13)^2$
Tack vare det kan vi ta alla okända termer ($x$-termer) på ena sidan i en ekvation och sedan ta kvadratroten ur båda led, vilket förvandlar andragradsekvationen till (oftast) två linjära ekvationer.

Andragradsuttryck där alla okända termer är samlade i en kvadrat kallas \emph{kvadratkompletterade}, och det är inte alla gånger som andragradsuttryck är skrivna på den formen.
I det här avsnittet ska vi lära oss att omvandla ett andragradsuttryck skrivet på allmän form ($ax^2 + bx + c$) till ett kvadratkompletterat uttryck.
Det arbetet kräver att vi är bekväma med att multiplicera parenteser som innehåller algebraiska uttryck, och att vi är säkra på att lösa linjära ekvationer.

\subsection{Att multiplicera två parentesuttryck}

I matte 1 har vi tränat på att multiplicera parenteser med ett tal, 
